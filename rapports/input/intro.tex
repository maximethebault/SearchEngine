\section{Introduction : Étude d’un moteur de recherche existant et évaluation comparative}

Ce document rend compte d’un travail effectué en 4è année du département informatique de l’INSA de Rennes dans le domaine de l’acquisition de connaissances, et plus spécifiquement dans la recherche d’informations (information retrieval).

Nous avons choisi d’étudier le fonctionnement d’un moteur de recherche relativement répandu et d’en faire une étude comparative. Il s’agit d’évaluer différentes configurations de ce moteur de recherche sur une collection de documents fournies par l’université de Glasgow (collection CISI) suivant différents critères.

Dans un premier temps, nous présentons Lucene, le moteur de recherche testé. Après avoir exposé des généralités sur le logiciel, nous donnons un bref aperçu de son fonctionnement, pour enfin nous concentrer sur les aspects pouvant être configurés. Nous abordons ainsi la notion d’analyseur, de termes et de filtres.

Dans un deuxième temps, nous décrivons la méthodologie utilisée pour l’étude comparative. On spécifie le processus de test et nous détaillons les critères utilisés pour l’évaluation.

S’en suit une présentation du programme réalisé. Il est chargé d’exécuter les différents tests et d’en retirer les critères évoqués précédemment. Nous évoquons dans cette partie les aspects les plus techniques de ce rapport, puisqu’il s’agit là de l’implémentation du banc de test tel qu’il aura été spécifié dans les parties précédentes.

Enfin, nous analysons les résultats des tests menés. Dans un premier temps, nous détaillons les configurations que nous avons choisi d’évaluer avant d’étudier dans un second temps les performances de chacune. Nous concluons cette partie avec une synthèse donnant les configurations optimales en fonction des critères retenus.

























