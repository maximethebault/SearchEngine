\section{Méthodologie de test}

Dans cette partie, nous présentons la méthodologie de test choisie. On détaille ainsi la collection utilisée pour les tests et les critères retenus pour chacune des deux grandes phases de la recherche, à savoir l’indexation et l’interrogation.

\subsection{Présentation de la collection CISI}
\label{section:presentationCISI}
Les recherches se font au travers d'une collection de documents nommée \textbf{CISI}. Cette collection est constituée d’un total de quatre fichiers, dont on va maintenant décrire le contenu et l’intérêt. On remarque que l’ensemble des données textuelles présentes dans ces fichiers est en anglais.

Un premier fichier, nommé \texttt{ALLnettoye}, contient un ensemble d'articles textuels (au total 1460). Un identifiant unique est associé à chaque article, ce qui permettra par la suite de les distinguer les uns des autres.

Un deuxième fichier, nommé \texttt{QRY}, contient un ensemble de requêtes (au total 112). Là encore, une requête est distinguée par un identifiant unique (non corrélé aux identifiants d’article cités précédemment). Dans le fichier, une requête revêt deux formes différentes : il s’agit soit d’un ensemble de questions (formulés dans un langage naturel), soit d’un article complet (le but étant de trouver les articles se rapprochant le plus de celui donné dans la requête). 

Un troisième fichier, nommé \texttt{motsvides}, contient une liste de mots vides. Il s’agit d’un ensemble de mots peu significatifs à ignorer lors de la recherche dans les articles, afin d’améliorer la pertinence des résultats.

Enfin, un dernier fichier, nommé \texttt{REL}, fait le lien entre les deux premiers fichiers. Il contient la liste des articles attendus comme résultats à l'exécution de chaque requête. Il prend la forme d'une liste d'associations \textbf{identifiant requête - identifiant article}, faisant correspondre une requête avec un des articles attendus en résultat. Cette liste a été réalisée par un ensemble d'experts et constitue donc les résultats théoriques. Ces résultats théoriques sont très importants pour notre étude comparative, ils nous permettent en effet d’estimer la qualité des résultats expérimentaux et sont utilisés pour établir deux des cinq critères retenus, que nous allons maintenant présenter.

\subsection{Critères d'indexation}

La phase d’indexation constitue la première étape de la recherche. Lors de cette phase, nous allons fournir à \textbf{Lucene} l’ensemble des documents de la collection \textbf{CISI}, à partir desquels il créera un index. Nous avons ainsi pu imaginer deux critères différents.

Le premier critère est le temps d’indexation. Il semble en effet pertinent de prendre en compte ce critère, certains indexes étant très dynamiques (nous pouvons par exemple imaginer les indexes des moteurs de recherche, qui sont constamment mis à jour pour prendre en compte les nouvelles pages web) : le temps d’indexation devient alors un élément crucial de performance. Ce critère peut cependant afficher des fluctuations importantes. En effet, l’indexation met en jeu des appels systèmes (notamment des appels au système de fichiers), dont le temps d’exécution dépend fortement de la charge de la machine. Pour mitiger ces fluctuations, l’indexation sera répétée un total de 55 fois : les durées des cinq premières indexations ne seront pas comptabilisées dans les résultats (temps de chauffe), et une moyenne sera calculée à partir des 50 autres itérations.

Le second critère retenu pour l’étape d’indexation est la taille de l’index. Après le coût en exécution, le coût en stockage est en effet le deuxième critère fondamental d’évaluation de complexité de programmes informatiques. Contrairement au premier critère, celui-ci ne connaît aucune fluctuation : \textbf{Lucene} produit toujours le même index à partir d’un même jeu de données (ce qui est plutôt rassurant). 

Après avoir établi ces 2 critères pour la phase d’indexation, passons maintenant à la phase d’interrogation.

\subsection{Critères d’interrogation}

Comme évoqué dans la sous-section \ref{section:presentationCISI}, nous allons utiliser les résultats théoriques de la collection \textbf{CISI}. Confrontés aux résultats pratiques, nous allons pouvoir définir deux critères : le rappel et la précision. Pour expliquer ces deux critères, il nous faut avant tout définir quatre notions : le vrai positif, le vrai négatif, le faux positif et le faux négatif.

Un \textbf{vrai positif (VP)} est un document présent dans les résultats théoriques comme dans les résultats pratiques. Un \textbf{vrai négatif (VN)} est un document absent des résultats théoriques, et qui n'est en effet pas présent pas les résultats pratiques. Un \textbf{faux positif (FP)} est un document non présent dans les résultats théoriques, mais qui est présent dans les résultats pratiques malgré tout. Enfin, un \textbf{faux négatif (FN)} est un document présent dans les résultats théoriques, mais qui est absent des résultats pratiques.

Le premier critère d'évaluation est le rappel, qui est un rapport entre le nombre de documents pertinents retournés et le nombre total de documents pertinents qui auraient dû être trouvés. La formule donnant le rappel est donc : \textbf{VP / (VP + FN)}. 

Le deuxième critère d'évaluation est la précision, qui est un rapport entre le nombre de documents pertinents retournés et le nombre total de documents retournés (pertinents ou pas). La formule donnant la précision est donc : \textbf{VP / (VP + FP)}.

Le rappel et la précision se calculent pour une requête donnée. Pour avoir des résultats tangibles, on calcule la moyenne de chacun de ces deux critères pour l’ensemble des 112 requêtes de la collection \textbf{CISI}.

Enfin, on utilise un troisième et dernier critère : le temps d’interrogation. Comme le temps d’indexation, le temps d’interrogation est soumis à d’importantes fluctuations. Il est donc déterminé avec plusieurs points de mesure (50 itérations, précédés de 5 itérations de chauffe).

On obtient ainsi les cinq critères sur lesquels notre étude comparative se base. Voyons maintenant l’implémentation technique qui permet le calcul de ces différents critères.